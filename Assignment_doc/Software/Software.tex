\documentclass{article}

\usepackage[margin=1in]{geometry} % Set margins to 1 inch
\usepackage{graphicx} % Allows including images
\usepackage{float} % Allows for precise placement of figures
\usepackage{amsmath} % Allows for math equations
\usepackage{siunitx} % Allows for SI units
\usepackage{placeins} % Makes sure images are in their respective sections by \FloatBarrier

\begin{document}

\title{Software Assignment Report\\ \large{Gargi Behera\\EE22BTECH11208}}
\author{}
\date{}
\maketitle

\maketitle

\section*{Objective :}
The objective of this assignment is to make a Python script that can make a playlist of songs and shuffle them randomly. The songs must be shuffled such that each song in the playlist is played before it gets looped.

\section*{Overview:}
\begin{itemize}
    \item Numpy library has been used, which allows us to randomize the songs
    \item We are playing the songs through GUI.
    \item Tkintker library has been used in the program to make the window.
    \item PyGame library has been used to play audio files.
    \item os module has been used to search file directory (cwd/songs) by default.
\end{itemize}

\section*{UI:}
\begin{figure}[ht]
	\centering
	\includegraphics[width=0.5\linewidth]{softUI.png}
	\caption{Screenshot of the UI}
	\label{fig:view}
\end{figure}
\FloatBarrier

\section*{Working:}
\begin{itemize}
    \item shuffle function in the program randomizes the order of the music files.
    \item Because the function only randomizes the order, there is no repetition of songs in the playlist.
    \item File managing is done completely through os module functions.
    \item Audio file playback is handled entirely through PyGame module functions. Pygame mixer is also used.
\end{itemize}

\subsection*{Shuffle function}
\begin{itemize}
    \item It replaces two elements with the second element to be replaced taken from randint function of numpy.random.
    \item As it replaces the elements, there is no repetition in the playlist.
    \item This function is executed whenever the playlist reaches the last song and user presses next song button.
\end{itemize}

\section*{Libraries used :}
\begin{enumerate}
    \item Pygame
    \item Random
    \item os
    \item tkinker
\end{enumerate}

\end{document}
