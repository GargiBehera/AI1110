\documentclass{article}

\usepackage[margin=1in]{geometry} % Set margins to 1 inch
\usepackage{graphicx} % Allows including images
\usepackage{float} % Allows for precise placement of figures
\usepackage{amsmath} % Allows for math equations
\usepackage{siunitx} % Allows for SI units
\usepackage{placeins} % Makes sure images are in their respective sections by \FloatBarrier

\begin{document}

\title{Hardware Assignment Report\\ \large{Gargi Behera\\EE22BTECH11208}}
\author{}
\date{}
\maketitle

\maketitle

\section*{Description}

\subsection*{Circuit Overview}
\begin{enumerate}
    \item Flip flops 7474 This device contains two independent positive-edge-triggered D-type flip-flops with complementary outputs.The information on the D input is accepted by the flip-flops on the positive going edge of the clock pulse.The Flip-flops take the clock from the clock bus and, based on their initial state, output a sequence of numbers.
    \item The sequence is fixed, and if the circuit is operated without concern for the initial state, the output number shown is generated randomly from 1 to 15 (both inclusive), with equal probability of all of them.
    \item The 7 Segment Decoder using IC 7447 is generally used as numerical indicators and consists of several LEDs arranged in seven segments. The decoder can show numbers from 0 to 15, and the ABCD formed by the flip-flops do not become 0000 at any time.
    \item This circuit is deterministic, hence, the randomness can be decoded by simply referring to the sequence.
    the sequence generated is 3,7,15,14,13,10,5,11,6,12,9,2,4,8,1,3,7...
    \item The output repeats after all 16 numbers are shown. 
\end{enumerate}
\subsubsection*{Timer}
\begin{enumerate}
    \item The time period of the display can be changed using different values of Resistor and Capacitor.
    \item A 10M$\Omega$ resistor and 47nF and 470nF capacitors are used in the project.
    \item This allows us to get a square pulse of 5V every 0.9 seconds. Which is slow enough to allow us to take readings from the resistor.
\end{enumerate}

 

\section*{Components}
\begin{enumerate}
    \item Breadboard
    \item Seven Segment Display - Common Anode
    \item 7447 Seven Segment Display Decoder
    \item 7474 D Flip-Flop x2
    \item 7486 XOR gate
    \item 555 precision timer
    \item Resistor 10M$\Omega$
    \item Resistor 1K$\Omega$
    \item Capacitor 47nF
    \item Capacitor 470nF
    \item USB micro B breakout board
    \item Jumper wires
\end{enumerate}

\section*{Practical Observations}
\begin{figure}[ht]
	\centering
	\includegraphics[width=1.0\linewidth]{"hwfig2.jpg"}
	\caption{Circuit}
	\label{fig:view}
\end{figure}
\FloatBarrier

\begin{figure}[ht]
	\centering
	\includegraphics[width=1.0\linewidth]{"hwfig3.jpg"}
	\caption{Circuit with power source}
	\label{fig:view}
\end{figure}
\FloatBarrier

\FloatBarrier

\section*{Block Diagram}

\begin{figure}[ht]
	\centering
	\includegraphics[width=0.7\linewidth]{"circuit.png"}
	\caption{Block Diagram}
	\label{fig:view}
\end{figure}
\FloatBarrier


\end{document}
